%! TEX root = **/010-main.tex
% vim: spell spelllang=en:

\section{Introduction}%
\label{sec:intro}
Space exploration is one of the great challenges humankind faces. Not only stemming
from our innate curiosity and eagerness to always know more, but also from the
ever-growing need to find other planets where our species can live. After all,
the human species is increasingly endangered due to climate change, among other
reasons, and while we might be able to overcome this dangers and the
threats might seem very small today, the risk of Earth becoming 
uninhabitable someday is real.

To pursue this goal, humans have built all kinds of telescopes that work from
land, such as National Radio Astronomy Observatory (NRAO) or the European
Northern and Southern Observatories, and from space, including the Hubble
Space Telescope. One of such telescopes is the Kepler Space Observatory. It was 
operational between May 2009 and October 2018, when it ran out of fuel.
During its time in service, it helped find over 2,600 exoplanets, which account
for around 60\% of all known exoplanets.

Data about all the potential exoplanets recorded by Kepler began to be published
in 2010. The dataset we used was published by NASA on Kaggle in October 2017.

Our goal for this project was to build a prediction model that could correctly 
classify candidate exoplanets as Confirmed or False Positives. To do so,
and to find the best possible model, we pre-processed the data and fed it to
four different machine learning
algorithms: Naive Bayes, K-NN, Decision Trees and Support Vector Machines. We
then compared the outcome of these algorithms to find which
works best for our dataset and target, and aimed to find the reasons
why the algorithms behaved the way they did.