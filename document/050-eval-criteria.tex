%! TEX root = **/010-main.tex
% vim: spell spelllang=en:

\section{Evaluation criteria of data mining models}%
\label{sec:eval-criteria}

% Description of the procedure followed in order to obtain a representative validation data set and
% description of the method that will be used to evaluate the different data mining models. 
% This includes description of parameters used in the evaluation (Cross-validation?
% K-fold cross-validation?  How many folders? Why that number?) 
% and discussion of the metric used for evaluation (accuracy,f1, etc).
% Which is the splitting procedure of data set into train and validation dataset? 
% Description of the procedure followed in order to obtain a representative validation data set.

To evaluate and validate our models, we used k-fold cross validation instead of
simply using cross validation, as we wanted to avoid the common problems associated
with it such as wasting data for testing or our models depending on a specific split.
Furthermore, it provides us with more robust estimators.

The value of k chosen for k-fold cross validation is 5. We selected it mainly
because of computational performance and execution time issues.


f1 score -> unbalanced dataset

The split of training and validation datasets was done 70:30 and we checked that the
target column maintained the proportions 

30%

Same seed

to optimize parameters we used k-fold cv

k-fold 5 most of the time, for performance issues but in some cases we increased that value