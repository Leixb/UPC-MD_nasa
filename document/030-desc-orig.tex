%! TEX root = **/010-main.tex
% vim: spell spelllang=en:

\section{Description of the original data}%
\label{sec:desc-orig}


% Where was it obtained (web page link), description of the problem on hands:target of classification,
% original number of examples, original number of columns meaning and kind of values,
% impact of missing values, etc.

\subsection{The dataset}

This dataset gathers close to 10,000 observations made by NASA's 
Kepler Space Observatory in pursuit of finding new exoplanets that
could potentially be habitable. 
We obtained the dataset from Kaggle\footnote{Kepler Exoplanet Search Results 
dataset on Kaggle:
\url{https://www.kaggle.com/nasa/kepler-exoplanet-search-results}},
the online data science and machine learning community.

9.564 observations were made by the Kepler observatory between 2009 and
2017, when this dataset was published, and all are contained in the dataset.
Moreover, it is comprised of 50 variables, most of them numerical and a few
categorical and boolean.

Regarding missing data, overall there are 40.557 values
missing, or what is the same 8.48\% of the dataset. However, if we take out the
three variables that are largely missing: \texttt{kepler\_name}
(76.01\% missing), \texttt{koi\_teq\_err1}(100\% missing) and 
\texttt{koi\_teq\_err2}(100\% missing), the missing data drops
to just under 3.15\%. Out of the remaining variables, most have a missing
data rate of around 3-4\% with the exception of \texttt{koi\_score}, which
stands shy of 16\%.

\subsection{Target Column}

In this dataset, the target of classification is quite clear:
\texttt{koi\_disposition}. This variable classifies the exoplanets into
four different categories:

    \paragraph{CANDIDATE} Observations that are still awaiting confirmation.
    
    \paragraph{FALSE POSITIVE} Observations that ended up not being Exoplanets.
    
    \paragraph{CONFIRMED} Exoplanet observations that have been confirmed.
    
    \paragraph{NOT DISPOSITIONED} There are no entries with this variable since the Kepler satellite was decommissioned
    in November 2019, and all the data has already been classified in the other categories.
    
We won't  consider CANDIDATE exoplanets, as we want to train and test our models
with examples we know for sure are (CONFIRMED) or aren't (FALSE POSITIVE) 
exoplanets.

\subsection{Column Description}

In Table \ref{tab:col_description}, we look at the dataset's variables: what
they mean, their type and what values they can take.

%\newcommand{\tnote}[1]{\textsuperscript{#1}}
\newcommand{\dd}{\textsuperscript{\ddag}}

\begin{xltabular}{\textwidth}{ l c X c }
    \caption{Column description \textsuperscript{\mathsection}}
    \label{tab:col_description} \\
    
    \toprule
    \textbf{Variable} & \textbf{Type\textsuperscript{\dagger}} & \textbf{Description} & \textbf{Values} \\
    \midrule
    \endfirsthead
    
    \toprule
    \textbf{Variable} & \textbf{Type\textsuperscript{\dagger}} & \textbf{Description} & \textbf{Values} \\
    \midrule
    \endhead
    
    \bottomrule
    \endfoot
    
    \bottomrule
    \endlastfoot
    
    \texttt{rowid} & N & Row number & [1-9564]\\
    
    \texttt{kepid} & N & Target identification number, as listed
    in the Kepler Input Catalog &  \\
    
    \texttt{kepoi\_name} & N & A number used to identify and track a Kepler Object of Interest (KOI) &  \\
    
    \texttt{kepler\_name} & N & Kepler number name identifying the planet &  \\
    
    \texttt{koi\_disposition} & CN & Category of the KOI from the Exoplanet Archive & \makecell{CANDIDATE\\ FALSE POSITIVE\\ CONFIRMED} \\
    
    \texttt{koi\_score} & N & 	A value between 0 and 1 that indicates the confidence in the KOI disposition & [0.0-1.0] \\
    
    \texttt{koi\_pdisposition} & CN & Pipeline flag that designates the most probable physical explanation of the KOI & \makecell{CANDIDATE\\ FALSE POSITIVE} \\
    
    \texttt{koi\_fpflag\_nt} & CN & Flag indicating the KOI has a light curve that is not consistent with that of a transiting planet & \makecell{0\\1} \\
    
    \texttt{koi\_fpflag\_ss} & CN & Flag indicating the KOI is observed to have a significant secondary event, transit shape, or out-of-eclipse variability & \makecell{0\\1} \\
    
    \texttt{koi\_fpflag\_co} & CN & Flag indicating that the source of the signal is from a nearby star & \makecell{0\\1} \\
    
    \texttt{koi\_fpflag\_ec} & CN & Flag indicating the KOI shares the same period and epoch as another KOI & \makecell{0\\1} \\
    
    \texttt{koi\_period}\dd & N & The interval between consecutive planetary transits &   \\
    
    \texttt{koi\_time0bk}\dd & N & The time corresponding to the center of the first detected transit in Barycentric Julian Day (BJD) minus a constant offset of 2,454,833.0 days &  \\
    
    \texttt{koi\_impact}\dd & N & The time corresponding to the center of the first detected transit in Barycentric Julian Day (BJD) minus a constant offset of 2,454,833.0 days &  \\
    
    \texttt{koi\_duration}\dd & N & The duration of the observed transits &  \\
    
    \texttt{koi\_depth}\dd & N & The fraction of stellar flux lost at the minimum of the planetary transit &  \\
    
    \texttt{koi\_prad}\dd & N & The radius of the planet &  \\
    
    \texttt{koi\_teq} & N & Approximation for the temperature of the planet &  \\
    
    \texttt{koi\_insol}\dd & N & Insolation flux i.e. another way to give the equilibrium temperature &   \\
    
    \texttt{koi\_model\_snr} & N & Transit depth normalized by the mean uncertainty in the flux during the transits &  \\
    
    \texttt{koi\_tce\_plnt\_num} & CN & TCE Planet Number federated to the KOI & [1-8]\\
    
    \texttt{koi\_tce\_delivname} & CN & TCE delivery name corresponding to the TCE data federated to the KOI & \makecell{q1\_q17\_dr25\_tce\\q1\_q16\_tce\\q1\_q17\_dr24\_tce} \\
    
    \texttt{koi\_steff}\dd & N & The photospheric temperature of the star &  \\
    
    \texttt{koi\_slogg}\dd & N & The base-10 logarithm of the acceleration due to gravity at the surface of the star &  \\
    
    \texttt{koi\_srad}\dd & N & The photospheric radius of the star &  \\
\end{xltabular}

\begin{itemize}[noitemsep,nolistsep]
  \item[\dagger] N = Numerical, CN = Categorical Nominal
  \item[\ddag] These variables have two corresponding columns named
  \texttt{koi\_VARNAME\_err1} and \texttt{koi\_VARNAME\_err2} that 
  represent the $+/-$ error respectively. We didn't include them in the table
  as they are repetitive and made it more confusing.
  \item[\mathsection] Descriptions obtained from NASA Kepler exoplanet KOI documentation \cite{noauthor_exoplanet_nodate}
\end{itemize}